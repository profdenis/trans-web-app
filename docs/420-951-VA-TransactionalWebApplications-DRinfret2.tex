% Options for packages loaded elsewhere
\PassOptionsToPackage{unicode}{hyperref}
\PassOptionsToPackage{hyphens}{url}
%
\documentclass[
  10pt,
]{article}
\usepackage{lmodern}
\usepackage{amssymb,amsmath}
\usepackage{ifxetex,ifluatex}
\ifnum 0\ifxetex 1\fi\ifluatex 1\fi=0 % if pdftex
  \usepackage[T1]{fontenc}
  \usepackage[utf8]{inputenc}
  \usepackage{textcomp} % provide euro and other symbols
\else % if luatex or xetex
  \usepackage{unicode-math}
  \defaultfontfeatures{Scale=MatchLowercase}
  \defaultfontfeatures[\rmfamily]{Ligatures=TeX,Scale=1}
  \setmainfont[]{Palatino}
  \setsansfont[]{Helvetica}
  \setmonofont[]{Menlo}
\fi
% Use upquote if available, for straight quotes in verbatim environments
\IfFileExists{upquote.sty}{\usepackage{upquote}}{}
\IfFileExists{microtype.sty}{% use microtype if available
  \usepackage[]{microtype}
  \UseMicrotypeSet[protrusion]{basicmath} % disable protrusion for tt fonts
}{}
\makeatletter
\@ifundefined{KOMAClassName}{% if non-KOMA class
  \IfFileExists{parskip.sty}{%
    \usepackage{parskip}
  }{% else
    \setlength{\parindent}{0pt}
    \setlength{\parskip}{6pt plus 2pt minus 1pt}}
}{% if KOMA class
  \KOMAoptions{parskip=half}}
\makeatother
\usepackage{xcolor}
\IfFileExists{xurl.sty}{\usepackage{xurl}}{} % add URL line breaks if available
\IfFileExists{bookmark.sty}{\usepackage{bookmark}}{\usepackage{hyperref}}
\hypersetup{
  hidelinks,
  pdfcreator={LaTeX via pandoc}}
\urlstyle{same} % disable monospaced font for URLs
\usepackage[margin=1.2in]{geometry}
\usepackage{longtable,booktabs}
% Correct order of tables after \paragraph or \subparagraph
\usepackage{etoolbox}
\makeatletter
\patchcmd\longtable{\par}{\if@noskipsec\mbox{}\fi\par}{}{}
\makeatother
% Allow footnotes in longtable head/foot
\IfFileExists{footnotehyper.sty}{\usepackage{footnotehyper}}{\usepackage{footnote}}
\makesavenoteenv{longtable}
\setlength{\emergencystretch}{3em} % prevent overfull lines
\providecommand{\tightlist}{%
  \setlength{\itemsep}{0pt}\setlength{\parskip}{0pt}}
\setcounter{secnumdepth}{5}


\author{}
\date{}

\begin{document}

\begin{center}
\hypertarget{vanier-college}{%
\section*{Vanier College}\label{vanier-college}}

\hypertarget{faculty-of-continuing-education}{%
\section*{Faculty of Continuing Education}\label{faculty-of-continuing-education}}
\end{center}

\begin{longtable}[]{@{}llll@{}}
\toprule
\endhead
\textbf{Course title} & Transactional Web Applications &
\textbf{Teacher} & Denis Rinfret\tabularnewline
\textbf{Course \#} & 420-951-VA & \textbf{Email} &
rinfretd@vaniercollege.qc.ca\tabularnewline
\textbf{Section} & 05813 & \textbf{Semester} & Nov.14'22 to
Feb.10'23\tabularnewline
\textbf{Schedule} & As shown on LÉA & \textbf{Room} & Online
(Zoom)\tabularnewline
\bottomrule
\end{longtable}

\hypertarget{course-description}{%
\subsection*{Course Description}\label{course-description}}

Web applications play an increasingly important role in society. In this
course, students design and build data-centric Web applications that
support information transactions for commercial, communication, and
other purposes. Using industry-standard Web-development technologies and
best practices, students are guided in the development, implementation,
and deployment of relevant Web applications.

\hypertarget{courses-role-in-the-program}{%
\subsection*{Course's role in the
program}\label{courses-role-in-the-program}}

This course completes the exploration of transactional Web applications
that has been started in the Front-End Web Programming course by
providing the tools to develop the code that resides on the Web
application server and that interacts with a database management system
or other information service as part of a complete transactional Web
application.

\hypertarget{statement-of-course-competencies}{%
\subsection*{Statement of course
competencies}\label{statement-of-course-competencies}}

The competency \textbf{00SU Develop transactional Web applications} will
be developed and finalized in this course. The competency was introduced
and developed in the Front-End Web Programming course.

\hypertarget{course-level-learning-outcome}{%
\subsection*{Course-level Learning
Outcome}\label{course-level-learning-outcome}}

Develop transactional Web applications.

\hypertarget{key-learning-outcomes}{%
\subsection*{Key Learning Outcomes}\label{key-learning-outcomes}}

Program transactional Web applications following the industry best
practices.

\begin{itemize}
\tightlist
\item
  Apply a widely used design pattern, such as MVC
\item
  Implement authentication and authorization
\item
  Apply appropriate security measures
\item
  Implement database transactions (CRUD)
\item
  Retrieve data from online sources, such as from Web APIs (JSON)
\item
  Apply testing procedures such as automated testing in a
  Behaviour-Driven Development methodology
\item
  Use proper collaboration tools for development and issue-tracking
\item
  Apply internationalization and localization techniques
\end{itemize}

\hypertarget{required-reading-material-textbook}{%
\subsection*{Required reading material
(Textbook)}\label{required-reading-material-textbook}}

There is no required textbook for this course. Lecture notes will be
provided through Omnivox, along with a list of recommended readings and
tutorials. The list will be updated throughout the course as needed.

\hypertarget{grading-scheme}{%
\subsection*{Grading Scheme}\label{grading-scheme}}

\begin{longtable}[]{@{}lll@{}}
\toprule
Component & Grade & Approx. due date\tabularnewline
\midrule
\endhead
Assignment 1 & 20\% & Dec.~12\tabularnewline
Assignment 2 & 20\% & Jan.~16\tabularnewline
Project (LIA) & 30\% & Feb.~10\tabularnewline
Exam & 30\% & Jan.~27\tabularnewline
\bottomrule
\end{longtable}


\hypertarget{late-work}{%
\subsubsection*{Late work}\label{late-work}}

A 10\% penalty for each late day will be applied on assignments
submitted after the deadline. The project (LIA) and the exam cannot be
submitted late. Exceptions can only be granted only according to the
College policies and procedures.

\hypertarget{project-learning-integration-assessment}{%
\subsection*{Project (Learning Integration
Assessment)}\label{project-learning-integration-assessment}}

Students work in groups to create, implement, and deploy a transactional
web application that incorporates the following features:

\begin{itemize}
\tightlist
\item
  User registration, authentication, and authorization, including
  two-factor authentication with QR code.
\item
  Prevention of common security risks such as SQL injection and
  cross-site scripting (XSS).
\item
  Implementing CRUD transactions
\item
  Internalization and localization
\item
  Reading data from online sources (Web APIs)
\item
  Developed using a Behaviour-Driven Development methodology, complete
  with a test suite automated with an appropriate framework, such as
  Behat.
\end{itemize}

Scope of the transactional web application:

\begin{itemize}
\item
  The scope should be representative of a typical small scale eCommerce
  application that can be developed by a team of two or three people.
\item
  Each student in this project implements ten user requirements, which
  are expressed as complete user stories written by the students
  themselves.
\item
  Upon completion, students present their web application and prepare a
  project report.
\end{itemize}

\hypertarget{evaluation-criteria-for-the-learning-integration-assessment}{%
\subsubsection*{Evaluation criteria for the learning integration
assessment}\label{evaluation-criteria-for-the-learning-integration-assessment}}

The following criteria are considered in the evaluation of the project
deliverables:

\begin{itemize}
\tightlist
\item
  In the implementation:

  \begin{itemize}
  \tightlist
  \item
    Respect of the selected design pattern (such as MVC).
  \item
    Correct validation of data inputs (form data inputs).
  \item
    Adequate corrective actions to ensure platform security.
  \item
    Reasonable consideration of User Experience.
  \item
    Appropriate design of User Interface.
  \item
    Appropriate test suite.
  \end{itemize}
\item
  In the documentation:

  \begin{itemize}
  \tightlist
  \item
    Correct presentation of work completed.
  \item
    Correct demonstration of product usability.
  \end{itemize}
\item
  In the presentation, clear communication of

  \begin{itemize}
  \tightlist
  \item
    Project objectives, need addressed, and target user-base.
  \item
    Difficulties met in the project and solutions.
  \item
    Lessons learned.
  \end{itemize}
\end{itemize}

\hypertarget{deliverables}{%
\subsubsection*{Deliverables}\label{deliverables}}

\begin{itemize}
\tightlist
\item
  Deliverable 1: Project proposal
\item
  Deliverable 2: HTML \& CSS
\item
  Deliverable 3: Project Implementation
\item
  Deliverable 4: Project report and presentation
\end{itemize}

\hypertarget{spli}{%
\subsubsection*{SPLI}\label{spli}}

The quality of English expression will be evaluated and be worth 10\% of
the total as part of grading the LIA written reports and in-class
presentations.

\hypertarget{teaching-methodology}{%
\subsection*{Teaching Methodology}\label{teaching-methodology}}

Programming exercises and discussion on sample programs can take place
either during the lecture time or lab period. All meetings will be held
\textbf{online} at the time of the scheduled classes using Zoom, through
the link provided on Omnivox.

\hypertarget{college-policies-procedures}{%
\subsection*{College policies \&
Procedures}\label{college-policies-procedures}}

There is a set of College policies and procedures covering the rights
and responsibilities of both faculty and students. These cover grade
review, student-faculty mediation, sexual harassment, standing and
advancement, cheating and plagiarism, absences for religious holidays.

Note that students who observe religious holidays during the semester
must inform the instructor, in writing, before the end of the first day
of online class. Consult \emph{Religious Holy Day Absences (see IPESA,
section 2.2.6)}. It is your responsibility to be aware of the various
policies and procedures governing your rights and obligations while you
are attending Vanier College.

Consult \emph{Student Academic Complaints (see 7210-8), Code of Conduct:
\url{http://www.vaniercollege.qc.ca/bylaws-policies-procedures/code-of-conduct/}
, Student Proficiency in the Language of Instruction (see 7210-33)} and
as well as any teacher or course-specific rules/guidelines that students
should adhere to (see Appendix 2).

\hypertarget{respecting-privacy-during-synchronous-online-classes}{%
\subsection*{Respecting Privacy during Synchronous Online
Classes}\label{respecting-privacy-during-synchronous-online-classes}}

The instructor might proceed for recording the student's image and voice
in the context of online synchronous lectures or labs class. Students
will be asked for their permission and agreement through form or MIO.
The recording will only be available to other students in the group
through and will be deleted once the block has ended.

\hypertarget{respecting-intellectual-property-rights-in-online-classes}{%
\subsection*{Respecting Intellectual Property Rights in Online
Classes}\label{respecting-intellectual-property-rights-in-online-classes}}

Any material produced as part of the course, including, but not limited
to, any pre-recorded or live video is protected by copyright,
intellectual property rights and image rights, regardless of the medium
used. It is strictly forbidden to record, copy,redistribute, reproduce,
republish, store in any way, retransmit or modify this material. Any
contravention of these conditions of use may be subject to sanction(s)
by the College under the Code of Conduct.

\hypertarget{cheating-plagiarism}{%
\subsection*{Cheating \& Plagiarism}\label{cheating-plagiarism}}

Any form of cheating or plagiarism will result in a grade zero for that
exam or assignment, and a letter from the course teacher will be placed
in your file. A repeated offence may lead to more serious consequences.
Consult The Vanier Student Writing Guide, the Vanier Catalogue, The
Student Handbook, Cheating and Plagiarism (see 7210-31), Student
Misconduct in the Classroom (7210-19) and your teacher for more
information.

\hypertarget{attendance-requirement}{%
\subsection*{Attendance requirement}\label{attendance-requirement}}

Students are responsible for material discussed during class time even
when they are absent. The material covered in class may be different
from what is presented in referenced material. Written and spoken class
material may be part of exams. There is no grade for attendance, but
students are responsible for in-class work and assessments whether they
are present or absent.

\hypertarget{weekly-breakdown-of-course-activities-tentative}{%
\subsection*{Weekly breakdown of course activities
(tentative)}\label{weekly-breakdown-of-course-activities-tentative}}

\begin{longtable}[]{@{}lll@{}}
\toprule
Week & Topics & Assessments\tabularnewline
\midrule
\endhead
1 & Course Introduction, Python &\tabularnewline
2, 3 & Flask basics, templates & Project proposal
(Dec.~5)\tabularnewline
4, 5 & Forms, sessions, authentication & Assignment 1
(Dec.~12)\tabularnewline
6, 7 & Database integration & Project: HTML \& CSS
(Dec.~9)\tabularnewline
8 & Security: 2FA, SQL injection, \ldots{} & Assignment 2
(Jan.~16)\tabularnewline
9 & Project work & Exam (Jan.~27)\tabularnewline
10, 11 & Project work & Project presentation \& submission
(Feb.~10)\tabularnewline
\bottomrule
\end{longtable}

\end{document}
